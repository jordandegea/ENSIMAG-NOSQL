\documentclass[a4paper,oneside,1pt]{article}
\usepackage[utf8]{inputenc}
\usepackage[T1]{fontenc} 
\usepackage{hyperref}
\usepackage{amsmath,amssymb}
\usepackage{fullpage}
\usepackage{graphicx}
\usepackage{url}
\usepackage{xspace}
\usepackage[french]{babel}
\usepackage{multicol}
\usepackage{geometry}

\usepackage[utf8]{inputenc}

\geometry{hmargin=2.5cm,vmargin=1.5cm}

\title{TP NoSQL - Neo4J}
\author{DANTIGNY Raynald - DE GEA Jordan - DUCLOT William}

\begin{document}

%You have to provide a compressed file that contains the report and two folders:
%- Folder 1 contains the data, scripts and instructions to populate the database.
%- Folder 2 contains the queries proposed (one file by query)

%The compressed file name must follow the next format: “noSQL_MSBigData_GroupXX”, where XX have to be changed by the corresponding number group assigned by Teide.

%Write a text describing the data stored in the database. This text should not exceed 15lines.You can include a figure to illustrate.
% Describe the indexes proposed and explain yourdecision.This text should not exceed 4linesby index.
% For each query, provide it :
%   in natural language.  This text should not exceed 3 lines
%   in the language of the systemand
%   If there are indexes involved, describe their impact. This text should not exceed 3linesby index
%   Advantages and disadvantages of the query implementation proposed.
% Text font size: 11 or 12


\maketitle

\section{Introduction}

\subsection{Choix technologique et sujet}

MongoDB est plus souvent utilisé que Neo4J, par exemple pour les backends mobile. Nous avons choisi d'utiliser Neo4j car les cas d'applications nous semblent plus intéressant et que ce TP est l'occasion de mettre en place une base Neo4j. \\
\linebreak

Notre sujet s'intitule "Prévision des listes gagnantes d'une école". Nous voulions appliquer ce TP à un cas concret. L'objectif est de faire des pronostics sur les résultats suite à une campagne. 

\section{Projet}

\subsection{Données}

% Diagramme

\subsection{Requêtes}

% Liste des requetes
% 1 file by query


\end{document}
