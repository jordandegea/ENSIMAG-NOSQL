\documentclass[a4paper,oneside,1pt]{article}
\usepackage[utf8]{inputenc}
\usepackage[T1]{fontenc} 
\usepackage{hyperref}
\usepackage{amsmath,amssymb}
\usepackage{fullpage}
\usepackage{graphicx}
\usepackage{url}
\usepackage{xspace}
\usepackage[french]{babel}
\usepackage{multicol}
\usepackage{geometry}

\usepackage[utf8]{inputenc}

\geometry{hmargin=2.5cm,vmargin=1.5cm}

\title{TP NoSQL - Neo4J}
\author{DANTIGNY Raynald - DE GEA Jordan - DUCLOT William}

\begin{document}

%You have to provide a compressed file that contains the report and two folders:
%- Folder 1 contains the data, scripts and instructions to populate the database.
%- Folder 2 contains the queries proposed (one file by query)

%The compressed file name must follow the next format: “noSQL_MSBigData_GroupXX”, where XX have to be changed by the corresponding number group assigned by Teide.

%Write a text describing the data stored in the database. This text should not exceed 15lines.You can include a figure to illustrate.

% Text font size: 11 or 12


\maketitle

\section{Introduction}

\subsection{Technology and Subject choice}

We knew that MongoDB was more used than Neo4j, e.g. for mobile backends. So we prefered to use Neo4J because this TP a chance for us to create and use a Neo4j database.  Furthermore, we chose Neo4j for applications which seems more interesting. 
\linebreak

Our subject is named "Prognosis of a school's winning lists". We wanted to apply this TP to a concrete application. The objective is to predict the results of lists after the campaign. In Bonus, we calculate costs and earnings for each lists. 

\subsection{How to use}

To populate the database, the file \texttt{generate.php} allow to generate random data. The output of this script should be redirected in a file (in case you don't have php, a data.txt file is already provided):

\texttt{php generate.php > data.txt}

The database can then be populated with this data: in the neo4j installation folder, execute the command

\texttt{./bin/neo4j-shell -file [path\_to\_data.txt] -path ./data/databases/graph.db}

\section{Project}

\subsection{Database}

% Diagramme

Each \textit{Liste} can propose several \textit{SOS}. \textit{SOS} is linked to one \textit{Liste}. \\
Each \textit{Liste} can propose several \textit{Event}. \textit{Event} is linked to one \textit{Liste} \\
Each \textit{Personne} can be in love with one \textit{Personne}. \\
Each \textit{Sponsor} can help one \textit{Liste}. \\
A defined number of \textit{Personne} (e.g. 25) can be in a \textit{Liste}. \\
Each \textit{Personne} can ask several \textit{SOS} from several \textit{Liste}. They can ask 0 or more times the same \textit{SOS}. \\
Each \textit{Personne} can participate in an \textit{Event}. \\
Each \textit{Personne} is in one \textit{Ecole}. \\
Each \textit{Liste} is in one \textit{Ecole}. \\
\subsection{Index}
\subsection{Index sur les noms}
Afin de sélectionner une personne par son nom plus rapidement on crée l'index suivant.
\\
CREATE INDEX ON :Personne(nom);
\subsection{Index sur les rôles}
Afin de sélectionner une personne listée par son rôle plus rapidement on crée l'index suivant.
\\
CREATE INDEX ON :EST\_LISTE(role);
\subsection{Request}

% Liste des requetes
% 1 file by query
% Describe the indexes proposed and explain yourdecision.This text should not exceed 4linesby index.
% For each query, provide it :
%   in natural language.  This text should not exceed 3 lines
%   in the language of the systemand
%   If there are indexes involved, describe their impact. This text should not exceed 3linesby index
%   Advantages and disadvantages of the query implementation proposed.

\subsection{Requete 0}
Cette requête permet de tester l'efficacité de l'indexation. On a donc indexé les personnes par leur nom.
\\
On veut sélectionner toutes les participation de la personne qui a pour nom "Nom14". 
\\
MATCH p = (:Personne{nom:"Nom14"})-[:PARTICIPE]-() RETURN p
\\L'indexation permet de directement sélectionner le noeud de la personne sans parcourir tous les noeuds de type Personne.

\subsection{Requete 1}
Retourner pour chaque liste son type (BDA, BDS, BDE), son président et son école.
\\
MATCH (l:Liste)-[:APPARTIENT]->(e:Ecole)
MATCH (p:Personne)-[:EST\_LISTE {role:'president'}]->(l)
RETURN l.nom,p.nom,e.nom

Pour accélérer la requête, on a créé un index sur le rôle de chaque listé, ainsi c'est plus rapide de trouver le président car il y en a qu'un seul et on a donc pas à parcourir tous les nœuds pour le trouver.


\subsection{Requete 2}
Calculer pour chaque liste la somme totale obtenue par ses sponsors
\\
On selectionne chaque relation liste/sponsor. Puis pour chaque liste, on additionne les montants des aides.
\\
MATCH (l:Liste)-[a:AIDE]-() 
RETURN l.nom, sum(a.montant) AS Subventions
\\
La requête est courte donc facilement lisible. Elle a en revanche le désavantage de ne pas avoir de clause GroupBy explicite comme en SQL.

\subsection{Requete 3}
Retourner les couples dont un des membres est dans plus d'une liste.
\\
Ici on prend d'abord tous les couples. Ensuite on regarde leur affiliation à des listes pour chacun des membres du couple. Enfin, on compte le nombre de liste dont est membre chaque personne du couple. On sélectionne finalement que les couples dont un des deux est listé à plus d'une liste.
\\
MATCH (p1:Personne)-[:EN\_COUPLE]->(p2:Personne)
MATCH (p1)-[:EST\_LISTE]->(l1:Liste)
MATCH (p2)-[:EST\_LISTE]->(l2:Liste)
WITH p1 as p1, p2 as p2, count(l2) as nombre\_listes2, count(l1) as nombre\_listes1
WHERE nombre\_listes1 > 1 OR nombre\_listes2 > 1 
RETURN p1.nom , p2.nom
ORDER BY p1.nom
\\
La requête à l'avantage d'être facilement compréhensible. Elle a pour désavantage d'afficher deux fois le couple car la relation de couple apparait deux fois pour les deux personnes.

\subsection{Requete 4}
Retourner les évènements par ordre d'appréciation proportionellement au nombre de participants.

On va récupérer l'appréciation associée à chaque participation a un évènement, sommer ces appréciations par évènement et diviser pas le nombre de participation à cet évènement.
\begin{verbatim}
MATCH (e:Evenement)<-[p:PARTICIPE]-(r:Personne)
RETURN e.nom, sum(p.appreciation)/count(p) AS score
ORDER BY score DESC
\end{verbatim}

\subsection{Requete 5}
Retourner la liste ayant effectué le plus de SOS
\\
On sélectionne d'abord tous les sos en gardant la relation avec la liste (qui servira pour trier par liste). Puis pour chaque sos, on sélectionne toutes les demandes de sos qu'il y a eu. On somme ensuite pour chaque liste, la somme des demandes de leur sos.
\\
MATCH (l:Liste)-[]->(s:SOS)
MATCH (s)-[demande:A\_DEMANDE]-()
RETURN l.nom, sum(demande.nombre) as nombre\_sos
ORDER BY nombre\_sos desc LIMIT 1
\\ 
Le GroupBy n'est toujours pas très explicite ici.

\subsubsection{Average opinion for each Event and each Liste} 

\subsubsection{List of Liste for each type and each Ecole}
\subsubsection{Finances for each Liste}
\subsubsection{Chance to win for each Liste}
\subsubsection{Number of demande for each SOS for each Liste}



\end{document}
