\documentclass[a4paper,oneside,1pt]{article}
\usepackage[utf8]{inputenc}
\usepackage[T1]{fontenc} 
\usepackage{hyperref}
\usepackage{amsmath,amssymb}
\usepackage{fullpage}
\usepackage{graphicx}
\usepackage{url}
\usepackage{xspace}
\usepackage[french]{babel}
\usepackage{multicol}
\usepackage{geometry}

\usepackage[utf8]{inputenc}

\geometry{hmargin=2.5cm,vmargin=1.5cm}

\title{TP NoSQL - Neo4J}
\author{DANTIGNY Raynald - DE GEA Jordan - DUCLOT William}

\begin{document}

%You have to provide a compressed file that contains the report and two folders:
%- Folder 1 contains the data, scripts and instructions to populate the database.
%- Folder 2 contains the queries proposed (one file by query)

%The compressed file name must follow the next format: “noSQL_MSBigData_GroupXX”, where XX have to be changed by the corresponding number group assigned by Teide.

%Write a text describing the data stored in the database. This text should not exceed 15lines.You can include a figure to illustrate.

% Text font size: 11 or 12


\maketitle

\section{Introduction}

\subsection{Technology and Subject choice}

We knew that MongoDB was more used than Neo4j, e.g. for mobile backends. So we prefered to use Neo4J because this TP a chance for us to create and use a Neo4j database.  Furthermore, we chose Neo4j for applications which seems more interesting. 
\linebreak

Our subject is named "Prognosis of a school's winning lists". We wanted to apply this TP to a concrete application. The objective is to predict the results of lists after the campaign. In Bonus, we calculate costs and earnings for each lists. 

\section{Project}

\subsection{Database}

% Diagramme

Each \textit{Liste} can propose several \textit{SOS}. \textit{SOS} is linked to one \textit{Liste}. \\
Each \textit{Liste} can propose several \textit{Event}. \textit{Event} is linked to one \textit{Liste} \\
Each \textit{Personne} can be in love with one \textit{Personne}. \\
Each \textit{Sponsor} can help one \textit{Liste}. \\
A defined number of \textit{Personne} (e.g. 25) can be in a \textit{Liste}. \\
Each \textit{Personne} can ask several \textit{SOS} from several \textit{Liste}. They can ask 0 or more times the same \textit{SOS}. \\
Each \textit{Personne} can participate in an \textit{Event}. \\
Each \textit{Personne} is in one \textit{Ecole}. \\
Each \textit{Liste} is in one \textit{Ecole}. \\

\subsection{Request}

% Liste des requetes
% 1 file by query
% Describe the indexes proposed and explain yourdecision.This text should not exceed 4linesby index.
% For each query, provide it :
%   in natural language.  This text should not exceed 3 lines
%   in the language of the systemand
%   If there are indexes involved, describe their impact. This text should not exceed 3linesby index
%   Advantages and disadvantages of the query implementation proposed.

\subsubsection{Average opinion for each Event and each Liste} 

\subsubsection{List of Liste for each type and each Ecole}
\subsubsection{Finances for each Liste}
\subsubsection{Chance to win for each Liste}
\subsubsection{Number of demande for each SOS for each Liste}



\end{document}
